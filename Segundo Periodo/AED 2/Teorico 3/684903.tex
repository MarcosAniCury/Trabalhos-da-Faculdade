Marcos Ani Cury Vinagre Silva-684903

	Somatórios:O que significa somatório, somatório é a soma de termos (n elementos), e em matemática é um operador da soma de termos de uma sequência, e usualente é utilizado a letra grega sigma (∑) para representar o somatório.
	Uma de suas aplicações é para expressar somas arbitrárias de números,
	Exemplo:
	      4	
   	     ∑ 2i, e esse somatório representa 2i desde 1 até 4. Logo teriamos 2.1 + 2.2 + 2.3 + 2.4.
	     i=1	
	O somatório é um elemento muito importante para a área da informatica pois, ela é muito utilizada para expressar o custo de um algoritmo de diversas formas, logo se torna algo essencial para quem pretende seguir a área, temos como exemplo da importancia de estudo do somatório no caso do caixeiro viajando, pois atráves do somatório pode-se perceber quanto tempo demora para executar o algoritmo e sem ele seria quase impossível calcular o tempo gasto pelo algoritmo sem sua implementação, logo podemos perceber a devida importancia dele.
	Além de utilizar o somatório para saber se o algoritmo é viável em questões de tempo ele também é importante para utilizarmos em questões de comparação e escolha do melhor algoritmo para cada caso, e temos como exemplo disso o quicksort que na maioria dos casos é a melhor estrutura de ordenação justamente pelo seu numero de operações, porém podemos avaliar atráves do somatório que no caso de uma estrutura ordenada ou quase ordenada é mais viável utilizar o algoritmo de inserção, e essa pequena diferença sútil entre os algoritmos pode ser percebida pela representação de seus numeros de operações que o somatório fornece.

Fontes:https://www.infopedia.pt/$somatorio, e os slides utilizados em aula.
		