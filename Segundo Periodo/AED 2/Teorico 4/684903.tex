Marcos Ani Cury Vinagre Silva - 684903

Teoria da complexidade e Classes de Problemas P,NP e NP-Completos

	A Teoria da complexidade computacional apartir de 1970 veio ganhando espaço dentre literaturas, artigos e livros de pessquisadores e estudiosos do assunto, e ele possui um enorme papel na área da computação. Numerosos problemas são considerados "intrátaveis" pois não existem algritmos "eficientes" que consigam resolve-los de forma ótima, da mesma forma que existem problemas parcialmente solucionaveis. Temos como exemplo de problemas "intrátaveis" os problemas de naturezacombinatória, pois o seu tempo de computação em função das enumerações das soluções viáveis, cresce muito rapidamente com o tamanho da entrada, porem em muitos casos é tão dificil prova-los "intrátaveis" quanto a descoberta de algoritmos "eficientes" para resolve-los.
	Temos como uma das classes de problemas, os problemas NP que são todos os problemas de busca (é um problema onde a solução proposta pode ser rapidamente verificada quantoo à correção), os problemas NP são categorizados como problemas de resolução em tempo polinomial ou exponencial e para todos os problemas de resolução em tempo polinomial classificamos como problemas de classe P, a nomenclatura P significa "polynomial time" e NP significa "nondeterministic polynomial time" e um algoritmo não determinpistico é um tipo especial de algoritmo (bastanto irreal) que "advinha corretamente" em todos os passos, assim NP são problemas cuja solução pode ser encontrada e vereficada em tempo polinomial por um algorimo não determinístico. Alguns pesquisadores acreditam na existencia de algoritmos de busca que não podem ser resolvidos, no entanto se mostrou extremamente difícil provar essa teoria e isso se tornou um dos enigmas mais profundos e importantes dos não resolvidos pela matemática.
	Inicialmente a definição de NP (e casualmente hoje em dia) era para problemas de decisão, e isso é defido a teoria do NP-completo. Um problema ele é classificado como NP-completo quando todos os problemas de busca se reduzem a ele (embora seja muito difícil) e esses notáveis problemas existem e alguns dos mais famosos são: Caixeiro Viajante, 3SAT, Longest Path, 3D Matching, Mochila, Independet Set, Zero-one Equations, Rudrata Path, Maximum Cut. Logo se conseguimos resolver de forma "eficiente" somente um deles, então iremos resolver todos os outros problemas NP de forma "eficiente".

Fontes:http://www.inf.ufpr.br/vignatti/courses/ci165/23.pdf, http://rmct.ime.eb.br/arquivos/RMCT_1_tri_1987/teoria_complex_comput.pdf, http://www2.fct.unesp.br/docentes/dmec/olivete/tc/arquivos/Aula10.pdf, http://www-di.inf.puc-rio.br/~hermann/slides-complexidade.pdf